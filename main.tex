% 서울대학교 전기·정보공학부 학사 학위논문 템플릿
% 논문 작성 지침에 일부 부합하지 않는 부분이 있을 수 있습니다.

\documentclass[10pt,a4paper]{report}
\usepackage{amssymb} % 수학
\usepackage{amsmath}
\usepackage{amsfonts}
\usepackage{siunitx} % SI단위
\usepackage{graphicx} % 그림
\usepackage{booktabs} % 표
\usepackage{indentfirst} % 첫 문장 들여쓰기

\usepackage[style=ieee]{biblatex} % IEEE형식 참고문헌
\addbibresource{bib.bib}

\usepackage{tikz} % 그림 그리기
\usetikzlibrary{positioning}

\usepackage[hangul]{kotex} % 한글
\usepackage[HWP]{dhucs-interword} % 아래아한글 자간
\usehangulfontspec{default}

\usepackage[list=off]{bicaption} % 이중 캡션
\captionsetup[table][bi-second]{name=Table}
\captionsetup[figure][bi-second]{name=Figure}
\usepackage{subcaption}

\def\papertitle{서울대학교 전기·정보공학부 졸업논문 템플릿} % 제목
\def\paperdate{2021년 8월} % 발간년도
\def\paperauthor{김철수} % 저자
\def\paperauthorspaced{김 철 수}
\def\paperadvisor{홍길동} % 지도교수
\def\paperkeywords{서울대학교, 공과대학, 전기·정보공학부, 졸업논문, 템플릿} % 국문 키워드
\def\paperenglishkeywords{SNU, ENG, ECE, thesis, template} % 국문 키워드

\usepackage[pdfauthor={\paperauthor},
            pdftitle={공학 학사 학위논문},
            pdfsubject={\papertitle},
            pdfkeywords={\paperkeywords, \paperenglishkeywords},
            ]{hyperref} % PDF

% 여백
\addtolength{\hoffset}{-1in}
\addtolength{\voffset}{-1in}
\setlength{\topmargin}{30mm}
\setlength{\headheight}{0mm}
\setlength{\headsep}{0mm}
\setlength{\marginparwidth}{0mm}
\setlength{\marginparsep}{0mm}
\setlength{\oddsidemargin}{25mm}
\setlength{\textwidth}{160mm}
\setlength{\textheight}{227mm}
\setlength{\footskip}{20mm}
\linespread{1.6} % double spacing 줄간격

% \chapter*도 목차에 추가
\makeatletter
\def\@makeschapterhead#1{%
  \addcontentsline{toc}{chapter}{#1}%
  {\centering\parindent \z@
    \normalfont
    \interlinepenalty\@M
    \Huge \bfseries  #1\par\nobreak
    \vskip 40\p@
  }}
\makeatother

\pagenumbering{alph} % 임시 알파벳 페이지 번호

\begin{document}
\renewcommand{\abstractname}{초~록}
\renewcommand{\contentsname}{목~차}
\renewcommand{\listtablename}{표~목차}
\renewcommand{\listfigurename}{그림~목차}

\thispagestyle{empty} % 외표지
\begin{center}
    \fontsize{16pt}{19pt}\selectfont
    공학 학사 학위논문 \\
    \vspace{2cm}
    \fontsize{21pt}{25pt}\selectfont
    \papertitle \\
    \vfill
    \fontsize{16pt}{19pt}\selectfont
    \paperdate \\
    \vspace{4cm}
    서울대학교 공과대학 \\
    \vspace{1cm}
    전기·정보공학부 \\
    \vspace{1cm}
    \paperauthorspaced
\end{center}
\newpage

\thispagestyle{empty} % 인준지
\begin{center}
    \fontsize{21pt}{25pt}\selectfont
    \papertitle \\
    \vspace{1cm}
    \fontsize{16pt}{19pt}\selectfont
    지도교수 \paperadvisor \\
    \vspace{1cm}
    이 논문을 공학 학사학위 논문으로 제출함. \\
    \vfill
    서울대학교 공과대학 \\
    \vspace{1cm}
    전기·정보공학부 \\
    \paperauthorspaced \\
    \vspace{1cm}
    \paperauthor의 학사 학위 논문을 인준함. \\
    \vspace{1cm}
    2021년 ~~~~~~월 ~~~~~~일 \\
    \vspace{1cm}
    지도교수 ~~~~~~~~~~~~~~~~~~~~~~~~~~~~~~~~(인)
\end{center}
\newpage

\pagenumbering{roman} % 로마 숫자 페이지 번호
\chapter*{\abstractname} % 초록

무덤 못 쉬이 말 것은 아무 있습니다. 시와 이름자 슬퍼하는 위에 다하지 별이 계십니다. 다 무덤 이름과, 못 아스라히 있습니다. 딴은 부끄러운 지나가는 시인의 사랑과 하나에 있습니다. 이름과, 하나에 이제 소학교 슬퍼하는 보고, 거외다. 별에도 언덕 별을 한 지나가는 북간도에 듯합니다. 벌써 밤을 가슴속에 경, 까닭입니다. 헤는 속의 어머님, 버리었습니다. 멀리 이름을 노새, 봅니다. 나의 청춘이 다 피어나듯이 내 이름자 경, 보고, 그리고 계십니다. 어머님, 하나에 이제 된 봅니다. 별 하나 가을로 위에 노루, 슬퍼하는 이 까닭입니다. 계절이 이름과 나의 옥 남은 가득 위에 있습니다. 이름과, 남은 내일 묻힌 내 속의 우는 한 듯합니다. 오는 헤일 하나에 릴케 사랑과 있습니다. 이름과 사람들의 별이 까닭입니다. 같이 까닭이요, 아침이 사랑과 거외다. 소녀들의 하나의 언덕 봄이 까닭입니다. 쉬이 잔디가 별 봅니다. 이 덮어 벌레는 이웃 둘 헤일 오는 당신은 거외다.
\vfill
주요어 : \paperkeywords

{ % 목차
\renewcommand\baselinestretch{1.3}
\tableofcontents
\listoftables
\listoffigures
}

\chapter{서론}\label{chap:introduction}
\pagenumbering{arabic} % 아라비아 숫자 페이지 번호

본 템플릿의 구성은 다음과 같다. \ref{chap:body}장 본론의 \ref{picture}절에서 그림의 예시를 보여준다. \ref{table}절에서 표의 예시를 보여준다. \ref{chap:conclusion}장에서는 본 템플릿을 요약한다.


\chapter{본론}\label{chap:body}

정보 엔트로피는 각 메시지에 포함된 정보의 기댓값으로 식~\eqref{eq:entropy}\과 같다\cite{6773024}.

\begin{equation}
    H(X) = -\sum_{i=1}^n {\mathrm{P}(x_i) \log_b \mathrm{P}(x_i)}\label{eq:entropy}
\end{equation}


\section{그림}\label{picture}

\begin{figure}[htp]
    \renewcommand\baselinestretch{1.3}
    \centering
    \begin{subfigure}[b]{0.5\textwidth}
        \centering
        \includegraphics[width=0.5\textwidth]{logo1.pdf}
        \bicaption{서울대학교 로고}{The logo of Seoul National University}
        \label{fig:snu}
    \end{subfigure}%
    \begin{subfigure}[b]{0.5\textwidth}
        \centering
        \includegraphics[width=0.9\textwidth]{logo2.pdf}
        \bicaption{공과대학 로고}{The logo of College of Engineering}
        \label{fig:eng}
    \end{subfigure}
    \bicaption[그림 예시 (목차 항목)]{그림 예시.}{An example of a figure.}
    \label{fig:example}
\end{figure}

그림 예시는 그림~\ref{fig:example}\와 같다. 그림~\ref{fig:snu}\은 서울대학교 로고이고 그림~\ref{fig:eng}\는 서울대학교 공과대학 로고이다.


\section{표}\label{table}

\begin{table}[htp]
    \renewcommand\baselinestretch{1.3}
    \centering
    \bicaption[표 예시 (목차 항목)]{표 예시.}{An example of a table.}
    \begin{tabular}{cc}
        \toprule
        상수 & 값 \\\midrule
        $c$ & \SI{299792458}{\meter\per\second} \\
        $h$ & \SI{6.62607015e-34}{\joule\per\hertz} \\\bottomrule
    \end{tabular}
    \label{tab:example}
\end{table}

표 예시는 표~\ref{tab:example}\과 같다.


\chapter{결론}\label{chap:conclusion}

무덤 못 쉬이 말 것은 아무 있습니다. 시와 이름자 슬퍼하는 위에 다하지 별이 계십니다. 다 무덤 이름과, 못 아스라히 있습니다. 딴은 부끄러운 지나가는 시인의 사랑과 하나에 있습니다. 이름과, 하나에 이제 소학교 슬퍼하는 보고, 거외다. 별에도 언덕 별을 한 지나가는 북간도에 듯합니다. 벌써 밤을 가슴속에 경, 까닭입니다. 헤는 속의 어머님, 버리었습니다. 멀리 이름을 노새, 봅니다. 나의 청춘이 다 피어나듯이 내 이름자 경, 보고, 그리고 계십니다. 어머님, 하나에 이제 된 봅니다. 별 하나 가을로 위에 노루, 슬퍼하는 이 까닭입니다. 계절이 이름과 나의 옥 남은 가득 위에 있습니다. 이름과, 남은 내일 묻힌 내 속의 우는 한 듯합니다. 오는 헤일 하나에 릴케 사랑과 있습니다. 이름과 사람들의 별이 까닭입니다. 같이 까닭이요, 아침이 사랑과 거외다. 소녀들의 하나의 언덕 봄이 까닭입니다. 쉬이 잔디가 별 봅니다. 이 덮어 벌레는 이웃 둘 헤일 오는 당신은 거외다.

\printbibliography

\chapter*{Abstract}

Lorem ipsum dolor sit amet, consectetur adipiscing elit. Nulla malesuada sit amet lacus eu ultricies. Fusce tempus, sem quis rhoncus efficitur, nisl erat tempus ligula, sit amet vulputate velit ante eu orci. Cras sodales lorem ac nisl fringilla rhoncus. Maecenas facilisis elit non venenatis eleifend. Nullam congue ligula molestie odio commodo, a vestibulum nibh lobortis. Vivamus id dignissim augue, non sagittis tortor. Donec id fermentum nibh. Suspendisse tortor nisl, cursus tincidunt urna eget, ornare venenatis sem. Aliquam iaculis rutrum tortor.

\vfill
Keywords: \paperenglishkeywords

\end{document}
